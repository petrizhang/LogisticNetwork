\documentclass{llncs}
\usepackage{amssymb}
\usepackage{graphicx}
\usepackage[ruled,linesnumbered,boxed]{algorithm2e}
\usepackage{graphicx}
\usepackage{amsmath}
%\usepackage{mathtools}
%\usepackage{color}
\usepackage{tabularx}
\usepackage[colorlinks, linkcolor=blue, anchorcolor=blue, citecolor=green]{hyperref}
%\usepackage{booktabs}
\usepackage[table]{xcolor}
%\uespackage{colortbl}
\usepackage[tight,footnotesize]{subfigure}
\usepackage{fancyhdr}
\usepackage{lastpage}
\usepackage{layout}
%\usepackage{ctex}

%\footskip = 10pt
\pagestyle{fancy}
\chead{Group Project}
\lhead{X033533-Algorithm@SJTU}
\rhead{Instructor: Xiaofeng Gao}
\rfoot{}
\cfoot{Page \thepage \ of \pageref{LastPage}}
\addtolength{\headheight}{0.5\baselineskip}
\addtolength{\headwidth}{0\marginparsep}
\addtolength{\headwidth}{0\marginparwidth}



\title{Logistic Network Design}
\subtitle{An ILP Approach \vspace{-3mm}}

\author{Author1 (ID, email), Author2 (ID, email), Author3 (ID, email)}
\institute{Department of Computer Science, \\ Shanghai Jiao Tong University, Shanghai, China}

\begin{document}
	\bibliographystyle{splncs}
	
	%\linespread{0.85}
	
	%==============================================================================
	\maketitle
	\begin{abstract}\vspace{-5mm}
		The abstract should briefly summarize the contents of the paper in
		15--250 words.
		
		\textbf{Keywords:} keywords 1, keywords2.
	\end{abstract}
	
\section{Preliminaries}
	In this section, we describe the variables and denotations we use and the assumptions we propose.
\subsection{Denotations}
	
	$\mathbf{L} = \{1,2,3...\}$: The cities set.
	
	$N = |\mathbf{L}|$: The total number of cities.
	
	$P = 2 * N * N$: The total number of all O-D pairs.
	
	$\mathbf{H} = \{1,2,3...\}$: The hubs set. 
	
	$D_{ij}$: Direct distance between city $i$ and $j$.
	
	$T_{ij}$: Direct travel time between city $i$ and $j$.
	
	$F_{ij}$: Flow demands from city $i$ to city $j$.
	
	$U_i$: Update cost for city $i$.
	
	$C_{ijkm}$: The transportation cost per unit flow from city $i$ to city $j$ routed from hub $k$ and $m$.

	$K_{i}$: The capacity of city $i$.
	
	$X_{ijkm}$ : The fraction of flow from origin $i$ to destination $j$ routed from hubs $k$ and $m$, in other words, the fraction of $F_{ij}$ through the path $i \rightarrow k \rightarrow m\rightarrow j$.
\subsection{Assumptions}
	
	We assume that the transportation cost per unit flow directly from city $p$ and city $q$ is is proportional to the direct distance between $p$ and $q$ and we denote the coefficient as $s$. Then we have:
	\begin{equation}
	C_{ijkm} = s(D_{ik}+\alpha D_{km}+D_{mj})
	\label{eq:c}
	\end{equation}
	
	\subsection{Decision variables}
	
	\begin{equation}
	X_{ijkm} \in [0,1]
	\end{equation}
	
	$X_{ijkm}$ is the fraction of flow from origin $i$ to destination $j$ routed from hubs $k$ and $m$, in other words, the fraction of $F_{ij}$ through the path $i \rightarrow k \rightarrow m\rightarrow j$.
	
	\begin{equation}\label{}
	Y_{i} = 
	\left\{  
	\begin{array}{ll}  
	1 & \text{where city $i$ is selected as a hub} \\  
	0 &  \text{where city $i$ is not selected as a hub}\\      
	\end{array}  
	\right.  
	\end{equation}
	
	\begin{equation}\label{}
	A_{ik} = 
	\left\{  
	\begin{array}{ll}  
	1 & \text{where city $i$ is allocated to hub $k$} \\  
	0 &  \text{where city $i$ is not allocated to hub $k$}\\      
	\end{array}  
	\right.  
	\end{equation}
	
\subsection{Object functions}
	The main aim of our model is to minimize the update cost, meanwhile the total transportation cost and delivery time are also in consideration but not important.
	
	Obviously, the \textbf{update cost} of all the hubs is:
	\begin{equation}
	    \sum_{k}U_k Y_{k}
	\end{equation}
	
	Then the \textbf{transportation cost} from $i$ to $j$ through network is:
	\begin{equation}
		\sum_{k}\sum_{m}F_{ij}  X_{ijkm} C_{ijkm}
	\end{equation}
	
	The \textbf{delivery time} from origin $i$ to destination $j$ through network is:
	\begin{equation}
	\sum_{k}\sum_{m}{(T_{ik}+ T_{km}+ T_{mj} ) A_{ik} A_{jm}}
	\end{equation}
	
	Then the total cost to build the entire network will be:
	\begin{equation}
		\sum_{i}\sum_{j}\sum_{k}\sum_{m}F_{ij}  X_{ijkm} C_{ijkm} + \sum_{k}U_k Y_{k}
	\end{equation}
	
	As the update cost is the main factor we consider, using the average transportation cost insead of total cost is more suitable. Then we get our first object function:
	
	\begin{equation}
		\frac{1}{P}\sum_{i}\sum_{j}\sum_{k}\sum_{m}F_{ij}  X_{ijkm} C_{ijkm} + \sum_{k}U_k Y_{k} \tag{\textbf{Object 1}}
	\end{equation}
	
	$P$ is the total number of all O-D pairs here.
	
	And the second object function is the average delivery time:
		\begin{equation}
		\frac{1}{P} \sum_{k}\sum_{m}{(T_{ik}+ T_{km}+ T_{mj} ) A_{ik} A_{jm}} \tag{\textbf{Object 2}}
	\end{equation}
	
\section{Uncapacitated single allocation formulation}
	In this section, we formulate the uncapacitade single allocation logistic network design as a ILP problem to minimize 2 objects:
	\begin{itemize}
		\item[1.] the cost including transportation cost and update cost .
		\item[2.] the average travel time between all O-D pairs.
	\end{itemize}
	
	
	\begin{alignat}{2}
	\min\quad
	& \frac{1}{P} \sum_{i}\sum_{j}\sum_{k}\sum_{m}F_{ij}  X_{ijkm} C_{ijkm} + \sum_{k}U_k Y_{k} \nonumber & &\\
	\quad& \frac{1}{P} \sum_{i}\sum_{j}\sum_{k}\sum_{m}{(T_{ik}+ T_{km}+ T_{mj} ) A_{ik} A_{jm}}  & & \tag{LP1}\\
	\mbox{s.t.}  \quad
	&Y_{k} \in \{0,1\} &\quad& \forall k \label{st1}\\ 
	&A_{ik} \in \{0,1\} &\quad& \forall i,k \label{st2}\\ 
	&A_{ik} \leq Y_{k} &\quad& \forall i,k \label{st3}\\
	&\sum_{k}{A_{ik}} = 1 &\quad& \forall i \label{st4}\\
	&\sum_{k}\sum_{m}{X_{ijkm} = 1}, &\quad& \forall i,j \label{st5}\\
	&X_{ijkm} \in \{0,1\} &\quad& \forall i,j,k,m \label{st6}\\
	&X_{ijkm} = A_{ik} * A_{jm} &\quad& \forall i,j,k,m \label{st7}
	\end{alignat}
	
	As Eq \ref{eq:c} shows, we have $ C_{ijkm} = s(D_{ik}+D_{km}+D_{mj})$.
	
	Eq \ref{st1},\ref{st2} assure that $Y_k$ and $A_{ik}$ are binary variables.
	
	Eq \ref{st3} enforces that when city $i$ is allocated to city $k$, city $k$ must be selected as a hub.
	
	Eq \ref{st4} enforces that one city $i$ must be allocated to a single hub $k$.
	
	Eq \ref{st5} enforces that all required flow between city $i$ and $j$ must be routed through proper hub $k$ and $m$, as the sum of the fractions of flow distributed to different paths is 1.
	
	Eq \ref{st6} sets $X_{ijkm}$ to binary variable, which means the flow between city $i$ and $j$ could only be routed through a singe path.
	
	Eq \ref{st7} enforces the route from city $i$ and $j$ through hub $k$ and $m$ is valid, in other words, city $k$ and $m$ must all be selected as hubs. \\

The decision variable $X_{ijkm}$ makes the model easier to extend to capacitated or multiple allocation problems, but in fact it is redundant here, as it would be either 0 or 1 and could be replaced with $X_{ijkm} = A_{ik} * A_{jm}$. Thus we could simplify the model as:
	\begin{alignat}{2}
		\min\quad
		& \frac{1}{P} \sum_{i}\sum_{j}\sum_{k}\sum_{m}F_{ij}  A_{ik} A_{jm} C_{ijkm} + \sum_{k}U_k Y_{k} & & \nonumber\\
		\quad& \frac{1}{P} \sum_{i}\sum_{j}\sum_{k}\sum_{m}{(T_{ik}+ T_{km}+ T_{mj} ) A_{ik} A_{jm}}  & & \tag{LP2} \label{lp2}\\
		\mbox{s.t.}  \quad
		&Y_{k} \in \{0,1\} &\quad& \forall k \label{st1.1}\\ 
		&A_{ik} \in \{0,1\} &\quad& \forall i,k \label{st1.2}\\ 
		&A_{ik} \leq Y_{k} &\quad& \forall i,k \label{st1.3}\\
		&\sum_{k}{A_{ik}} = 1 &\quad& \forall i \label{st1.4}
	\end{alignat}


\subsection{Solution}

	This is a multi-objective programming problem, there are 2 common approaches to solve this kind of problems:
	\begin{itemize}
		\item[1.] Focus on the first object function and solve the problem, get the optimal value $opt$. Then set the value as a up bound and find the optimal value for for the second object function.
		\item[2.] Simply add the 2 object function with proper weights $w_1$ and $w_2$, and to optimize the new object function $w_1*OBJ1+w_2*OBJ2$.
	\end{itemize}
	
	We prefer the first approach.
	
\section{Uncapacitated multiple allocation formulation}
	Similarly to the single allocation problem, we just modify the constrains of $A_{ik}$ and $X_{ijkm}$ to model the multiple allocation problem. Obviously, the flow demands from city $i$ and $j$ could be distributed to multiple different paths here.
	
	\begin{alignat}{2}
	\min\quad
	& \frac{1}{P} \sum_{i}\sum_{j}\sum_{k}\sum_{m}F_{ij}  X_{ijkm} C_{ijkm} + \sum_{k}U_k Y_{k} & & \nonumber\\
	\quad& \frac{1}{P} \sum_{i}\sum_{j}\sum_{k}\sum_{m}{(T_{ik}+ T_{km}+ T_{mj} ) A_{ik} A_{jm}}  & & \tag{LP3} \label{lp3}\\
	\mbox{s.t.}  \quad
	&Y_{k} \in \{0,1\} &\quad& \forall k \label{st2.1}\\ 
	&A_{ik} \in \{0,1\} &\quad& \forall i,k \label{st2.2}\\ 
	&A_{ik} \leq Y_{k} &\quad& \forall i,k \label{st2.3}\\
	&\sum_{k}\sum_{m}{X_{ijkm} = 1}, &\quad& \forall i,j \label{st2.4}\\
	& 0 \leq X_{ijkm} \leq 1 &\quad& \forall i,j,k,m \label{st2.5}\\
	&X_{ijkm} \leq A_{ik} * A_{jm} &\quad& \forall i,j,k,m \label{st2.6}
	\end{alignat}
	
	
	Comparing the single allocation formulation, we make the following changes: 
	\begin{itemize}
		\item[1.] Remove the constraint $\sum_{k}{A_{ik}} = 1, \forall i$ to assure one city $i$ could be allocated to multiple hubs.
		\item[2.] $X_{ijkm}$ represents the flow distributed fractions of city $i$ to city $j$, we modify it from a binary variable 0 or 1 to a decimal number in range of $[0,1]$.
		\item[3.] Eq \ref{st2.6} means that, if there are some flows routed through the path $i\rightarrow k \rightarrow m \rightarrow j$, then city $i$ must be allocated to hub $k$ and city $j$ must be allocated hub $m$.
	\end{itemize}

\section{Capacitated single allocation formulation}	
	We simply add a capacity constraint for the sum of all valid paths routed through hub $m$. 
	\begin{alignat}{2}
		\min\quad
		& \frac{1}{P} \sum_{i}\sum_{j}\sum_{k}\sum_{m}F_{ij}  X_{ijkm} C_{ijkm} + \sum_{k}U_k Y_{k} & & \nonumber\\
		\quad& \frac{1}{P} \sum_{i}\sum_{j}\sum_{k}\sum_{m}{(T_{ik}+ T_{km}+ T_{mj} ) A_{ik} A_{jm}}  & & \tag{LP4}\label{lp4}\\
		\mbox{s.t.}  \quad
		&Y_{k} \in \{0,1\} &\quad& \forall k \label{st3.1}\\ 
		&A_{ik} \in \{0,1\} &\quad& \forall i,k \label{st3.2}\\ 
		&A_{ik} \leq Y_{k} &\quad& \forall i,k \label{st3.3}\\
		&\sum_{k}{A_{ik}} = 1 &\quad& \forall i \label{st3.4}\\
		&\sum_{k}\sum_{m}{X_{ijkm} = 1}, &\quad& \forall i,j \label{st3.5}\\
		&X_{ijkm} \in \{0,1\} &\quad& \forall i,j,k,m \label{st3.6}\\
		&X_{ijkm} = A_{ik} * A_{jm} &\quad& \forall i,j,k,m \label{st3.7} \\
		&\sum_{i}\sum_{j}\sum_{k}{F_{ij}X_{ijkm} \leq K_{m}} &\quad& \forall m \label{st3.8} 
	\end{alignat}

\section{Capacitated multiple allocation formulation}	
Similarly to the capacitated single allocation formulation, we add a capacity constraint for the sum of all valid paths routed through hub $m$.
	\begin{alignat}{2}
		\min\quad
		& \frac{1}{P} \sum_{i}\sum_{j}\sum_{k}\sum_{m}F_{ij}  X_{ijkm} C_{ijkm} + \sum_{k}U_k Y_{k} & & \nonumber\\
		\quad& \frac{1}{P} \sum_{i}\sum_{j}\sum_{k}\sum_{m}{(T_{ik}+ T_{km}+ T_{mj} ) A_{ik} A_{jm}}  & & \tag{LP5}\label{lp5}\\
		\mbox{s.t.}  \quad
		&Y_{k} \in \{0,1\} &\quad& \forall k \label{st4.1}\\ 
		&A_{ik} \in \{0,1\} &\quad& \forall i,k \label{st4.2}\\ 
		&A_{ik} \leq Y_{k} &\quad& \forall i,k \label{st4.3}\\
		&\sum_{k}\sum_{m}{X_{ijkm} = 1}, &\quad& \forall i,j \label{st4.4}\\
		& 0 \leq X_{ijkm} \leq 1 &\quad& \forall i,j,k,m \label{st4.5}\\
		&X_{ijkm} \leq A_{ik} * A_{jm} &\quad& \forall i,j,k,m \label{st4.6} \\
		&\sum_{i}\sum_{j}\sum_{k}{F_{ij}X_{ijkm} \leq K_{m}} &\quad& \forall m \label{st3.8}
	\end{alignat}

\section{n-hub allocation approximation algorithm}

We fix the hubs number to $n$ before network design to decrease the problem complexity and propose a approximation algorithm to solve this kind of n-hub allocation problem.

Intuitively, the distance between any 2 hubs $k$ and $m$ should in an acceptable range, which menas the value of $d_{km}$ could not be too large or too small. Thus we indtroduce 2 distance threshholds $\tau_{ld}$ and $\tau_{hd}$. For every 2 hubs $k$ and $m$, the distance between them must satisfy $\tau_{ld} \leq d_{km} \leq \tau_{hd}$. This constraint assure our algorithm could terminate.

For a city $i$ and two hubs $k$ and $m$,  we introduce 2 metrics $hc(i,k,m)$ and $ht(i,k, m)$ to measure the transportation cost and travel time through the path $i \rightarrow k \rightarrow m$. 
\begin{equation}
	hc(i,k,m) = s(F_{ik}D_{ik} + \alpha F_{km}D_{km})
\end{equation}
\begin{equation}
    ht(i,k,m) = T_{ik} + T_{km}
\end{equation}
Here $s$ is the coefficient between transportation cost and flow-distance product and $\alpha$ is the transportation cost discount factor between hubs and non-hubs.


To optimize both the 2 values, we simply add them with weights $w_1$ and $w_2$ to get the final determine metric:
\begin{equation}
hscore(i,k,m) = w_1 \cdot hc(i,k, m) + w_2 \cdot ht(i,k, m)
\end{equation}

We focus on following 2 conditions to find a reasonable good hub allocation:

\begin{itemize}
	\item[1.] $n$ hubs $\mathbf{Hubs}= \{h_1,h_2,...,h_n\}$ are determined and we need to assign the owned hub for a free city $i$.
	
	In this condition, we introduce $city\text{-}hub\text{-}score$ to determine the best hub for city $i$. For a city $i$ and a hub $k$, $city\text{-}hub\text{-}score(i,k, \mathbf{Hubs})$ is the sum of $hscore$ from the city $i$ to every hub $m$ through hub $k$:
	\begin{equation}
	city\text{-}hub\text{-}score(i,k,\mathbf{Hubs}) = \sum_{m \in \mathbf{Hubs}} hscore(i,k,m)  
	\end{equation}
	
	City $i$ will be allocated to the hub $k$ which has the highest $city\text{-}hub\text{-}score$.
	
	\item[2.] $n-1$ hubs $\mathbf{Hubs}= \{h_1,h_2,...,h_{-1}\}$ are determined and we need to find the last hub in the
	rest cities $\mathbf{Cities}=\{city_1,city_2,...,city_t\}$ to form a complete network.
	
	In this condition, to determine a city $k$ if should be selected as a hub, we introduce the metric $hub\text{-}score(k,\mathbf{Hubs},\mathbf{Cities})$ to measure the choice:
	\begin{equation}
		hub\text{-}score(k,\mathbf{Hubs},\mathbf{Cities})) = \sum_{i\in \mathbf{Cities}} \sum_{m \in \mathbf{Hubs}} hscore(i,k,m)  
	\end{equation}
	
	For specific city $k$, the metric means the total score of the paths $\text{any city} \rightarrow k \rightarrow \text{all hubs}$.
\end{itemize}

Contructed on $city\text{-}hub\text{-}score$ and $hub\text{-}score$, the basic idea of our algorithm is:

\begin{itemize}
	\item [1.] Randomly choose $n$ hubs at the begining. 
	\item [2.] Calculate the $city\text{-}hub\text{-}score$ for all cities and then allocate these cities to hubs based on this score.
	\item [3.]  Travel the hubs set, fro every hub $i$ and the cities it dominate, remove $i$ from hubs set and mark these cities as free cities, calculate the  $hub\text{-}score$ for all no-hub cities(including the cities owned by other hubs) then select a new hub for the free cities. Put the new hub to hubs set.
	\item [4.] If the new hubs set equals to the previous one, then algorithm terminates.
	\item[5.] Repeat from step 2 until the distance between every 2 hubs satisfies the threshold constraint $\tau_{ld} \leq d_{km} \leq \tau_{hd}$ or the algorithm reaches step 4.
\end{itemize}
\begin{minipage}[t]{\textwidth}
	\begin{algorithm}[H]
		\BlankLine
		\SetKwInOut{Input}{input}
		\SetKwInOut{Output}{output}
		$\mathbf{Cities} \leftarrow \{1,2,...,N\}$\;
		$Owner[1..N] \leftarrow$ Init an array represents the owner hub of cities\;
		$\mathbf{Hubs} \leftarrow \text{Randomly select $n$ hubs}$\;
		\caption{n-hub($n$, $\tau_{ld}$, $\tau_{hd}$, $F[1..N][1..N]$, $D[1..N][1..N]$, $T[1..N][1..N]$)}\label{Alg.n-hub}
		\While{true}{
			\For{$i \in \mathbf{Cities}$ and $i \notin \mathbf{Hubs}$} {
				$bestowner \leftarrow \mathop{\arg\min}_{k}{city\text{-}hub\text{-}score(i,k,\mathbf{Hubs})}$ forall $k \in \mathbf{Hubs}$\; 
				$owner[i] \leftarrow bestowner$\;
			}
			$\mathbf{OldHubs} \leftarrow \mathbf{Hubs}$\;
			\For{$h \in \mathbf{Hubs}$} {
				$\mathbf{Hubs} \leftarrow \mathbf{Hubs} \setminus \{h\}$\;
				$\mathbf{FreeCities} \leftarrow CitiesOwnedBy(h) \cup \{h\}$\;
				$betterhub \leftarrow \mathop{\arg\max}_{k}{hub\text{-}score(k,\mathbf{Hubs}, \mathbf{FreeCities}))}$ forall $k\in \mathbf{Cities}$\;
				$\mathbf{Hubs} \leftarrow \mathbf{Hubs} \cup \{betterhub\}$\;
				$Owner[betterhub] \leftarrow betterhub$\;
				\For{$i \in \mathbf{Cities}$}{
					$Owner[i] \leftarrow betterhub$\;
				}

		  }
	  		
		  \If{$\mathbf{Hubs} \neq \mathbf{OldHubs}$ or $\tau_{ld} \leq d_{km} \leq \tau_{hd}, \forall k,m \in \mathbf{Hubs} $} {
		  	\textbf{break}\;
		  }
		}		
	\end{algorithm}
\end{minipage}

For $N$ cities, to find a k-hub allocation, the complexity of our algorithm is $O(ck^2N^2)$, here $c$ is the total rounds our algorithm execute. Cenerally $c$ would be a very small constant. The parameter $\tau_{ld}$ and $\tau_{hd}$ will decrease the execute time our this algorithm. We could also set them to $-\infty$ and $\infty$ to for better accuracy. In our experiments we just ignore these 2 parameters because the performance of the program is in an acceptable range.

\section{Evalution}
We evalute our \ref{lp2} model on the given dataset.
To assure the transportation cost and the update cost are in the same order of magnitude, we set $s=0.00001$. Here $N=81$ then $P = 2*N*N = 13122$. 

2-stage muli-obejctive optimizition costs two much time, thus we simply add the two object functions to get the final object function :

\begin{align}
	 \quad & \frac{1}{P} \sum_{i}\sum_{j}\sum_{k}\sum_{m}F_{ij}  A_{ik} A_{jm} C_{ijkm} + \sum_{k}U_k Y_{k} 
	 + \frac{1}{P} \sum_{i}\sum_{j}\sum_{k}\sum_{m}{(T_{ik}+ T_{km}+ T_{mj} ) A_{ik} A_{jm}} \nonumber \\
	 = \quad &   \sum_{k}U_k Y_{k} \nonumber \\
	 & + \frac{s}{P} \sum_{i}\sum_{j}\sum_{k}\sum_{m}F_{ij}  A_{ik} A_{jm} (D_{ik}+D_{km}+D_{mj}) \nonumber\\
     & + \frac{1}{P} \sum_{i}\sum_{j}\sum_{k}\sum_{m}{(T_{ik}+ T_{km}+ T_{mj} ) A_{ik} A_{jm}}	 \nonumber\\
     = \quad &\sum_{k}U_k Y_{k} \nonumber\\
     & + \frac{0.00001}{13122} \sum_{i}\sum_{j}\sum_{k}\sum_{m}F_{ij}  A_{ik} A_{jm} (D_{ik}+D_{km}+D_{mj})   \nonumber\\
     & + \frac{1}{13122} \sum_{i}\sum_{j}\sum_{k}\sum_{m}{(T_{ik}+ T_{km}+ T_{mj} ) A_{ik} A_{jm}}
\end{align}

The constraints are same with \ref{lp2}:
\begin{alignat}{2}
	\mbox{s.t.}  \quad
	&Y_{k} \in \{0,1\} &\quad& \forall k \label{st1.1}\\ 
	&A_{ik} \in \{0,1\} &\quad& \forall i,k \label{st1.2}\\ 
	&A_{ik} \leq Y_{k} &\quad& \forall i,k \label{st1.3}\\
	&\sum_{k}{A_{ik}} = 1 &\quad& \forall i \label{st1.4}
\end{alignat}

We build this model with OPL language and solve the problem in IBM CPLEX, 


	
	\section*{Acknowledgements}
	
	Here is your acknowledgements. You may also include your feelings, suggestion, and comments in the acknowledgement section.
	
	%
	% ---- Bibliography ----
	%
	% BibTeX users should specify bibliography style 'splncs04'.
	% References will then be sorted and formatted in the correct style.
	%
	% \bibliographystyle{splncs04}
	% \bibliography{mybibliography}
	%
	\begin{thebibliography}{8}
		\bibitem{ref_article1}
		Author, F.: Article title. Journal \textbf{2}(5), 99--110 (2016)
		
		\bibitem{ref_lncs1}
		Author, F., Author, S.: Title of a proceedings paper. In: Editor,
		F., Editor, S. (eds.) CONFERENCE 2016, LNCS, vol. 9999, pp. 1--13.
		Springer, Heidelberg (2016).
		
		\bibitem{ref_book1}
		Author, F., Author, S., Author, T.: Book title. 2nd edn. Publisher,
		Location (1999)
		
		\bibitem{ref_proc1}
		Author, A.-B.: Contribution title. In: 9th International Proceedings
		on Proceedings, pp. 1--2. Publisher, Location (2010)
		
		\bibitem{ref_url1}
		LNCS Homepage, \url{http://www.springer.com/lncs}. Last accessed 4
		Oct 2017
	\end{thebibliography}
\end{document} 